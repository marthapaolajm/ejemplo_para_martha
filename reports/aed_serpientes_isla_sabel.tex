\documentclass{article}
\usepackage{csvsimple}
\usepackage{float}
\usepackage{booktabs}
\usepackage{pgfplotstable}
\usepackage[a4paper,top=1cm,bottom=2cm,left=3cm,right=3cm,marginparwidth=1.75cm]{geometry} 
\pgfplotstableread[col sep=comma]{tables/resumen_cinco_numero_serpientes.csv}\ResumenCincoNumeros

\author{Martha Paola Jiménez Martínez}
\title{Análisis Explotario de Datos en Isla Isabel}

\begin{document} 
\section{Introducción}
\section{Análisis Exploratorio de Datos}
\section{Resumen de cinco números}
\section{Resultados}
\begin{table}
\centering
\caption{Ejemplo}
\label{tab:csvResumenCincoNumeros}
\csvautotabular{tables/resumen_cinco_numero_serpientes.csv}
\end{table}

\begin{table}[H]
   \centering
   \caption{Ejemplo contenido de tabla}
    \pgfplotstabletypeset[
      string type,
      assign column name/.style={/pgfplots/table/column name={\textbf{#1}}},
        every head row/.style={before row={\toprule
          & \multicolumn{3}{c}{\textbf{Longitudes (cm)}} & \multicolumn{3}{c}{}\\
          }, after row=\midrule},
          every last row/.style={after row=\bottomrule
        },
     ]{\ResumenCincoNumeros}
   \label{tab:ResumenCincoNumeros}
 \end{table}

\end{document}
