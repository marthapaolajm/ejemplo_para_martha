\documentclass{article}
\usepackage{csvsimple}
\usepackage{float}
\usepackage{booktabs}
\usepackage{pgfplotstable}
\usepackage[a4paper,top=1cm,bottom=2cm,left=3cm,right=3cm,marginparwidth=1.75cm]{geometry} 
\pgfplotstableread[col sep=comma]{tables/resumen_cinco_numero_serpientes.csv}\ResumenCincoNumeros

\author{Martha Paola Jiménez Martínez}
\title{Análisis Exploratorio de Datos en Isla Isabel}

\begin{document} 
\section{Introducción}
El análisis exploratorio de datos, pretende obtener información de los datos en Isla Isabel de la
serpiente falsa coralillo (2008-2014). Y este incluye, dos elementos: Disponibilidad de datos; el
cual es representado mediante una tabla que contiene el nombre de la categoría, desviación estándar,
resumen de cinco números, promedio y cifra de datos. Así también, contiene una visualización de
resumen; escenificado por un diagrama de cajas.

\section{Análisis Exploratorio de Datos}
El resumen de cinco números está constituido por: la desviación estándar, el mínimo, máximo y los cuartiles uno, dos y tres. Donde el cuartil uno $(Q_1)$ representa el 25\%, el cuartil dos $(Q_2)$ representa el 50\% y el cuartil tres $(Q_3)$ representa el 75\% de los datos. 
El diagrama de cajas nos sirve para visualizar los datos atípicos. Cada caja está delimitada por los cuartiles uno y tres, donde la línea en el centro de las caja; es la mediana (cuartil dos). Y los puntos fuera de los diagramas de cajas, son los datos atípicos.

\begin{figure}
\centering
\caption{Boxsplot}
\label{fig:pngBoxsplotLongitudTotal}
\includegraphics{figures/diagrama_cajas_serpientes_isabel.png}
\end{figure}

\begin{table}[H]
   \centering
   \caption{Ejemplo contenido de tabla}
    \pgfplotstabletypeset[
      string type,
      assign column name/.style={/pgfplots/table/column name={\textbf{#1}}},
        every head row/.style={before row={\toprule
          & \multicolumn{3}{c}{\textbf{Longitudes (cm)}} & \multicolumn{3}{c}{}\\
          }, after row=\midrule},
          every last row/.style={after row=\bottomrule
        },
     ]{\ResumenCincoNumeros}
   \label{tab:ResumenCincoNumeros}
 \end{table}

\end{document}
