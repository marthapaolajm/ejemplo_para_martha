\documentclass{article}
\usepackage{csvsimple}
\usepackage{float}
\usepackage{booktabs}
\usepackage{pgfplotstable}
\usepackage{pythontex}
\usepackage{hyperref}
\usepackage[a4paper,top=1cm,bottom=2cm,left=3cm,right=3cm,marginparwidth=1.75cm]{geometry} 
\pgfplotstableread[col sep=comma]{tables/resumen_cinco_numero_serpientes.csv}\ResumenCincoNumeros

\author{Martha Paola Jiménez Martínez}
\title{Análisis Exploratorio de Datos en Isla Isabel}

\begin{pycode}
import pythontex_tools as ptt

datapackage = '../data/raw/datapackage.json'

metadata_writer = ptt.Writer_Metadata()
metadata_writer.load_metadata(datapackage)

\end{pycode}

\begin{document} 
\maketitle


\section{Datos}
Usamos la \py{metadata_writer.write_title_with_link()}
\py{metadata_writer.description()}

El análisis exploratorio de datos en Isla Isabel de la serpiente falsa coralillo (2008-2014).
Incluye la disponibilidad de datos; el cual es representado mediante una tabla que contiene el
nombre de la categoría, desviación estándar, resumen de cinco números, promedio y cifra de datos.
Así también, contiene una visualización de resumen; escenificado por un diagrama de cajas.

\section{Análisis Exploratorio de Datos}
El resumen de cinco números está constituido por: la desviación estándar, el mínimo, máximo y los cuartiles uno, dos y tres. Donde el cuartil uno $(Q_1)$ representa el 25\%, el cuartil dos $(Q_2)$ representa el 50\% y el cuartil tres $(Q_3)$ representa el 75\% de los datos. 
El diagrama de cajas nos sirve para visualizar los datos atípicos. Cada caja está delimitada por $(Q_1)$ y $(Q_3)$, donde la línea en el centro de las caja; es la mediana $(Q_1)$. Y los puntos fuera de los diagramas de cajas, son los datos atípicos.

\begin{table}[H]
   \centering
   \caption{Resumen de cinco números}
    \pgfplotstabletypeset[
      string type,
      assign column name/.style={/pgfplots/table/column name={\textbf{#1}}},
        every head row/.style={before row={\toprule
          & \multicolumn{3}{c}{\textbf{Longitudes (cm)}} & \multicolumn{3}{c}{}\\
          }, after row=\midrule},
          every last row/.style={after row=\bottomrule
        },
     ]{\ResumenCincoNumeros}
   \label{tab:ResumenCincoNumeros}
 \end{table}

\section{Resultados}

En la Figura 1 podemos observar las alteraciones por temporada de la longitud cloaca cola.

\begin{figure}[H]
\centering
\caption{Las temporadas 2008, 2009 y 2013 tienen un dato atípico cada uno y en el 2014 no hay cuartil tres.}
\label{fig:pngBoxsplotLongitudTotal}
\includegraphics[scale=0.4]{figures/diagrama_cajas_Longitud_cloaca_cola_serpientes_isabel.png}
\end{figure}

En la Figura 2 podemos observar las alteraciones por temporada de la longitud hocico cola.

\begin{figure}[H]
\centering
\caption{De las temporadas 2008 y 2009 se tienen diversos datos atípicos y en el 2011 se aprecia que su mediana, es más cercana al cuartil tres que las demás temporadas.}
\label{fig:pngBoxsplotLongitudTotal}
\includegraphics[scale=0.4]{figures/diagrama_cajas_Longitud_hocico_cola_serpientes_isabel.png}
\end{figure}

En la Figura 3 podemos observar las alteraciones por temporada de la longitud total.

\begin{figure}[H]
\centering
\caption{En las temporadas 2008 y 2009 se obtienen datos atípicos, así también en 2008 y 2011 ambas medianas con más cercanas al cuartil tres.}
\label{fig:pngBoxsplotLongitudTotal}
\includegraphics[scale=0.4]{figures/diagrama_cajas_Longitud_total_serpientes_isabel.png}
\end{figure}

En la Figura 4 podemos observar las alteraciones por temporada de la masa del individuo.

\begin{figure}[H]
\centering
\caption{En las temporadas 2008 y 2013 se conservan datos atípicos, sin embargo también se puede observar que en 2013 no hay un máximo en los datos.}
\label{fig:pngBoxsplotLongitudTotal}
\includegraphics[scale=0.4]{figures/diagrama_cajas_Masa_del_individuo_serpientes_isabel.png}
\end{figure}

\end{document}
