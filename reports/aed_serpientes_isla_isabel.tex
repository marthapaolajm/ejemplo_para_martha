\documentclass{article}
\usepackage{csvsimple}
\usepackage{float}
\usepackage{booktabs}
\usepackage{pgfplotstable}
\usepackage{pythontex}
\usepackage{hyperref}
\usepackage[spanish,es-tabla,es-nodecimaldot]{babel}
\usepackage{multirow}
\usepackage[a4paper,top=1cm,bottom=2cm,left=3cm,right=3cm,marginparwidth=1.75cm]{geometry} 
\pgfplotstableread[col sep=comma]{tables/resumen_cinco_numero_serpientes.csv}\ResumenCincoNumeros

\begin{document} 

\begin{pycode}
import pythontex_tools as ptt

datapackage = '../data/raw/datapackage.json'

metadata_writer = ptt.Writer_Metadata() 
metadata_writer.load_metadata(datapackage)

\end{pycode}

\author{Martha Paola Jiménez Martínez \\ 
\small{martha.jimenez@islas.org.mx}} \title{Análisis Exploratorio de Datos en Isla Isabel \\
\begin{large} Grupo de Ecología y Conservación de Islas \end{large}}  

\maketitle

\begin{abstract}
Presentamos un análisis exploratorio de datos de las medidas morfométricas, de serpiente falsa
coralillo en Isla  Isabel en el periodo de 2008 al 2014. Calculamos un resumen de cinco números, y
utilizamos diagramas de cajas para visualizar los datos, donde podemos observar que los datos
colectados fueron consistentes para las diferentes temporadas.
\end{abstract}

\section*{Datos}
Usamos la \py{metadata_writer.write_title_with_link()}
\py{metadata_writer.description()}

\section*{Análisis Exploratorio de Datos}
Mostramos un análisis exploratorio de los datos de serpiente falsa coralillo en Isla Isabel en el
periodo de 2008 al 2014. Incluimos la disponibilidad de datos; la cual esta representada mediante la
tabla \ref{tab:ResumenCincoNumeros} que contiene: cantidad de datos, resumen de cinco números y
promedio.

El resumen de cinco números está constituido por los siguientes estadísticos: la desviación estándar
($\sigma$), el mínimo, máximo y los cuartiles uno, dos y tres. Agrupando los datos por medida
morfométrica y por temporada, creamos diferentes conjuntos de datos $\delta_n$. El cuartil uno
$(Q_1)$ representa el 25\% del conjunto, el cuartil dos $(Q_2)$ representa el 50\% o la mediana y el
cuartil tres $(Q_3)$ representa el 75\%.  Para complementar, mostramos una visualización del
resumen; escenificado por un diagrama de cajas.  El diagrama de cajas nos sirve para visualizar los
estadísticos de la tabla \ref{tab:ResumenCincoNumeros} y los datos atípicos.  Cada caja está
delimitada por $(Q_1)$ y $(Q_3)$, donde la línea naranja de las cajas; es la mediana $(Q_2)$. Y los
puntos fuera de los diagramas de cajas, son los datos atípicos.



\begin{table}[H]
   \centering
\caption{Resumen de cinco números de medidas morfometricas de serpiente falsa coralillo
($Lamprompeltis triangulum$) en Isla Isabel  con datos del 2008 a 2014. Donde $N_{\delta}$ es la
canidad de registros, $\bar{\delta}$ es el promedio, $\sigma$ es la desviacion estandar, min es el
minimo, max es el maximo y $Q_1$, $Q_2$ y $Q_3$ son los cuartines uno, dos y tres respectivamente.}
\vspace{0.5cm}
\pgfplotstabletypeset[string type, assign column name/.style={/pgfplots/table/column
name={\textbf{#1}}}, every head row/.style={before row={\toprule}, after row=\midrule}, every last
row/.style={after row=\bottomrule},]{\ResumenCincoNumeros}
   \label{tab:ResumenCincoNumeros}
 \end{table}

\section*{Resultados}

\subsection*{Longitud cloaca cola}
En la Figura \ref{fig:pngBoxsplotLongitudCloacaCola} podemos observar un diagrama de cajas por
temporada de la longitud cloaca cola, de los individuos de falsa coralillo capturados en la Isla
Isabel con datos del 2008 a 2014. Comparando los diagramas de las siete temporadas, no se observa
mucha variacion en la mediana.

\begin{figure}[H]
\centering
\label{fig:pngBoxsplotLongitudCloacaCola}
\includegraphics[scale=0.4]{figures/diagrama_cajas_Longitud_cloaca_cola_serpientes_isabel.png}
\caption{Diagrama de cajas por temporada de la longitud cloaca cola, de los individuos de falsa coralillo capturados en la Isla Isabel con datos del 2008 a 2014. En las temporadas 2008, 2009 y 2013 capturamos algunos datos atípicos.}
\end{figure}

\subsection*{Longitud hocico cola}
En la Figura \ref{fig:pngBoxsplotLongitudHocicoCola} podemos observar un diagrama de cajas por
temporada de la longitud hocico cola, de los individuos de falsa coralillo capturados en la Isla
Isabel con datos del 2008 a 2014. Comparando los diagramas de las siete temporadas, se observa  que
las medianas son consistente.

\begin{figure}[H]
\centering
\label{fig:pngBoxsplotLongitudHocicoCola}
\includegraphics[scale=0.4]{figures/diagrama_cajas_Longitud_hocico_cola_serpientes_isabel.png}
\caption{Diagrama de cajas por temporada de la longitud hocico cola, de los individuos de falsa coralillo capturados en la Isla Isabel con datos del 2008 a 2014. En las temporadas 2008 y 2009 se tienen diversos datos atípicos, posiblemente a que se capturaron individuos de diferentes edades. En el 2011 se aprecia que su mediana, es más cercana al cuartil tres que las demás temporadas.}
\end{figure}

\subsection*{Longitud total}
En la Figura \ref{fig:pngBoxsplotLongitudTotal} podemos observar un diagrama de cajas por temporada
de la longitud total, de los individuos de falsa coralillo capturados en la Isla Isabel con datos
del 2008 a 2011. Se observa que las medianas de las cuatro temporadas, son consistentes.

\begin{figure}[H]
\centering
\label{fig:pngBoxsplotLongitudTotal}
\includegraphics[scale=0.4]{figures/diagrama_cajas_Longitud_total_serpientes_isabel.png}
\caption{Diagrama de cajas por temporada de la longitud total, de los individuos de falsa coralillo capturados en la Isla Isabel con datos del 2008 a 2011. En las temporadas 2008 y 2009 se obtienen datos atípicos, así también en 2008 y 2011 ambas medianas con más cercanas al cuartil tres.}
\end{figure}

\subsection*{Masa}
En la Figura \ref{fig:pngBoxsplotMasa} podemos observar un diagrama de cajas por temporada de la
masa, de los individuos de falsa coralillo capturados en la Isla Isabel con datos del 2008 a 2014.
Comparando los diagramas de las siete temporadas, se observa  que la variacion de las medianas no es
consistente.

\begin{figure}[H]
\centering
\label{fig:pngBoxsplotMasa}
\includegraphics[scale=0.4]{figures/diagrama_cajas_Masa_del_individuo_serpientes_isabel.png}
\caption{Diagrama de cajas por temporada de la masa, de los individuos de falsa coralillo capturados en la Isla Isabel con datos del 2008 a 2014. En las temporadas 2008 y 2013 se conservan datos atípicos.}
\end{figure}

\end{document}
