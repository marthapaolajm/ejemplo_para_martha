\documentclass[12pt,letterpaper]{article}
\usepackage[utf8]{inputenc}
\usepackage[spanish]{babel}
\usepackage{amsmath}
\usepackage{amsfonts}
\usepackage{amssymb}
\usepackage{makeidx}
\usepackage{graphicx}
\usepackage{lmodern}
\usepackage{hyperref}
\usepackage{natbib}
\usepackage[left=2.5cm,right=2.5cm,top=2.5cm,bottom=2.5cm]{geometry}
\thispagestyle{empty}
\begin{document}
%%%%%%%%%%%%%%%%%%%%%%%%%%
\title{\textbf{Tarea 1: Herramientas para Ciencia de Datos.}} \author{Martha Paola Jiménez
Martínez \\
\small{martha.jimenez@islas.org.mx}}
\date{ \small{Asignatura: PVVC 2021-1}
} 
\maketitle

%%%%%%%%%%%%%%%%%%%%%%%%%%%%%%%%%%%%%%%%%%%%%%

\section*{Introducción}
La ciencia de datos, explora datos para obtener conocimiento de ellos; lo que implica saber programación, estadística, comunicación, etc. Y para poder aplicar estos conocimientos, es esencial el dominio de diversas plataformas, algunas de ellas se mostrarán a continuación con su significado y su utilidad, así como algunos conceptos que forman parte de la ejecución de la ciencia de datos.

\section*{Herramientas}
\subsection*{Git}
Git es una herramienta que realiza una función del control de versiones de
código de forma distribuida, y no depende de un repositorio central. Este
funciona con ramas las cuales están destinadas a provocar proyectos divergentes
de un proyecto principal, para hacer experimentos o para probar nuevas
funcionalidades.
\citep{rubio_2020}
Sirve para que la copia de trabajo del código de cada desarrollador sea también
un repositorio que puede albergar el historial completo de todos los cambios.
\citep{bitbucket_git}

\subsection*{Bash}
Bash es un intérprete de comandos que ejecuta, una por una, las instrucciones
introducidas por el usuario o contenidas en un script y devuelve los resultados.
En otras palabras, actúa como interfaz entre el kernel Linux y los usuarios o
programas del modo texto.
\citep{canepa_2018}

\subsection*{Docker}
Es una tecnología de creación de contenedores que permite la creación y el uso
de contenedores de Linux. Sirve para usar los contenedores como máquinas
virtuales extremadamente livianas y modulares. Además, obtiene flexibilidad con
estos contenedores: puede crearlos, implementarlos, copiarlos y moverlos de un
entorno a otro, lo cual le permite optimizar sus aplicaciones para la nube.
\citep{redhat_docker}
También nos ayuda a asegurar la reproducibilidad de los resultados y no tener conflictos con las dependencias.

\subsection*{GitHub}
GitHub es un sistema de gestión de proyectos y control de versiones de código,
así como uno de los repositorios online más grandes de trabajo colaborativo en
todo el mundo.
\citep{hostinger_tutoriales_2019}
Sirve como un servicio de hosting de repositorios almacenados en la nube.
Esencialmente, hace que sea más fácil para los individuos y equipos usar Git
como la versión de control y colaboración.
\citep{kinsta_2020}

\subsection*{GitFlow}
Es un flujo de trabajo aplicado a un repositorio Git. Sirve para proyectos que
lleven una planificación de entregas iterativas. Permite la paralelización del
desarrollo mediante ramas independientes para la preparación, mantenimiento y
publicación de versiones del proyecto así como soporta la reparación de errores
en cualquier momento.
\citep{claventy_2020}

\subsection*{Python}
Python es un lenguaje interpretado, es decir, que no conlleva compilar
(scripting), independiente de plataforma y orientado a objetos. 
\citep{desarrollo_web_2003}
Sirve para interpretar el código del programador, donde a su vez lo traduce y
ejecuta. Y utiliza diversos modelos de desarrollo, por lo que no exige un estilo
único de programación.
\citep{angeles_2020}

\subsection*{Jupyter Notebook}
Es una interfaz web de código abierto que permite la inclusión de texto, vídeo,
audio, imágenes así como la ejecución de código a través del navegador en
múltiples lenguajes. 
\citep{cabrera_diaz_jupyter}
Se utiliza para: 
\begin{itemize}
\item Depuración de datos: distinguir entre los datos que son importantes y los
que no lo son al ejecutar un análisis de big data.
\item Modelización estadística: método matemático para estimar la probabilidad
de distribución de una característica concreta.
\item Creación y entrenamiento de modelos de aprendizaje automático: diseño,
programación y entrenamiento de modelos basados en aprendizaje automático
\item Visualización de datos: representación gráfica de datos para visualizar
con claridad patrones, tendencias, interdependencias, etc.
\citep{digital_guide_ionos_jupyter}
\end{itemize}

\subsection*{Análisis exploratorio de datos}
Ayudan a organizar la información que nos dan los datos de manera de detectar
algún patrón de comportamiento así como también apartamentos importantes al
modelo subyacente.
\citep{orella_analisis_exploratorio}
Su objetivo es explorar, describir, resumir y visualizar la naturaleza de los
datos recogidos en las variables aleatorias del proyecto o investigación de
interés, mediante la aplicación de técnicas simples de resumen de datos y
métodos gráficos sin asumir asunciones para su interpretación.
\citep{heix_bios_analisis_exploratorio}

\subsection*{Curación de datos}
Consiste en seleccionar informaci\'on relevante que ya está publicada en
Internet, filtrarla, organizarla, añadir un valor adicional y difundirla a
nuestra comunidad.
\citep{duro_2017}
Su función es hacer que la información sea útil y tomar las medidas necesarias
para preservarla.
\citep{it_user_2020}

\section*{Cuentas}
\begin{itemize}
\item LastPass: Es una extensión de un navegador, que almacena las contraseñas de forma encriptada, lo que te permite compartirlas sin la necesidad de mostrarlas.
\item GitHub: Es el alma de Git, con la que se puede interactuar y colaborar con diversos usuarios. Y almacena todo en la nube.
\item GitKraken: Es el cerebro de Git, donde podemos organizar nuestros repositorios y decidir que cambio queremos hacer antes de nuestra acción final.
\item Gravatar (WordPress): Es una imagen reconocida globalmente, que te sigue por cada sitio web donde aparezca tu correo electrónico.
\item Trello: Es una interfaz que permite la organización de proyectos.
\item Bitbucket: Es un servicio que aloja proyectos, que al igual que GitHub permite la colaboración entre usuarios.
\end{itemize}

\bibliography{../references/herramientas} 
\bibliographystyle{apalike}

\end{document}
