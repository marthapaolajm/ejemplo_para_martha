\documentclass[12pt,letterpaper]{article}
\usepackage[utf8]{inputenc}
\usepackage[spanish]{babel}
\usepackage{amsmath}
\usepackage{amsfonts}
\usepackage{amssymb}
\usepackage{makeidx}
\usepackage{graphicx}
\usepackage{lmodern}
\usepackage{hyperref}
\usepackage{natbib}
\usepackage[left=2.5cm,right=2.5cm,top=2.5cm,bottom=2.5cm]{geometry}
\thispagestyle{empty}
\begin{document}
%%%%%%%%%%%%%%%%%%%%%%%%%%
\title{\textbf{Tarea 0: 
Definiciones.}}
\author{Martha Paola Jim\'enez Mart\'inez \\
\small{martha.paola.jimenez.martinez@uabc.edu.mx} \\
\small{martha.jimenez@islas.org.mx}}
\date{ \small{Asignatura: PVVC 2021-1}
} 
\maketitle

%%%%%%%%%%%%%%%%%%%%%%%%%%%%%%%%%%%%%%%%%%%%%%
\section*{Herramientas}
\subsection*{Git}
Git es una herramienta que realiza una funci\'on del control de versiones de c\'odigo de forma distribuida, y no depende de un repositorio central.
Este funciona con ramas las cuales est\'an destinadas a provocar proyectos divergentes de un proyecto principal, para hacer experimentos o para probar nuevas funcionalidades.
\citep{GitRubio}
Sirve para que la copia de trabajo del c\'odigo de cada desarrollador sea tambi\'en un repositorio que puede albergar el historial completo de todos los cambios.
\citep{BitAtlas}

\subsection*{Bash}
Bash es un int\'erprete de comandos que ejecuta, una por una, las instrucciones introducidas por el usuario o contenidas en un script y devuelve los resultados. En otras palabras, act\'ua como interfaz entre el kernel Linux y los usuarios o programas del modo texto.
\citep{CanBlog}

\subsection*{Docker}
Es una tecnolog\'ia de creación de contenedores que permite la creaci\'on y el uso de contenedores de Linux.
Sirve para usar los contenedores como m\'aquinas virtuales extremadamente livianas y modulares. Adem\'as, obtiene flexibilidad con estos contenedores: puede crearlos, implementarlos, copiarlos y moverlos de un entorno a otro, lo cual le permite optimizar sus aplicaciones para la nube.
\citep{RedDocker}

\subsection*{GitHub}
GitHub es un sistema de gesti\'on de proyectos y control de versiones de c\'odigo, as\'i como uno de los repositorios online m\'as grandes de trabajo colaborativo en todo el mundo.
\citep{HostGitHub}
Sirve como un servicio de hosting de repositorios almacenados en la nube. Esencialmente, hace que sea m\'as f\'acil para los individuos y equipos usar Git como la versi\'on de control y colaboraci\'on.
\citep{KinGitHub}

\subsection*{GitFlow}
Es un flujo de trabajo aplicado a un repositorio Git. 
Sirve para proyectos que lleven una planificaci\'on de entregas iterativas. Permite la paralelización del desarrollo mediante ramas independientes para la preparaci\'on, mantenimiento y publicaci\'on de versiones del proyecto as\'i como soporta la reparaci\'on de errores en cualquier momento.
\citep{ClaGit}

\subsection*{Python}
Python es un lenguaje interpretado, es decir, que no conlleva compilar (scripting), independiente de plataforma y orientado a objetos. 
\citep{DesPy}
Sirve para interpretar el c\'odigo del programador, donde a su vez lo traduce y ejecuta. Y utiliza diversos modelos de desarrollo, por lo que no exige un estilo \'unico de programaci\'on.
\citep{AnPy}


\subsection*{Jupyter Notebook}
Es una interfaz web de c\'odigo abierto que permite la inclusi\'on de texto, v\'ideo, audio, im\'agenes as\'i como la ejecuci\'on de c\'odigo a trav\'es del navegador en m\'ultiples lenguajes. 
\citep{CaDiaJupyter}
Se utiliza para: 
\begin{itemize}
\item Depuraci\'on de datos: distinguir entre los datos que son importantes y los que no lo son al ejecutar un an\'alisis de big data.
\item Modelizaci\'on estad\'istica: m\'etodo matem\'atico para estimar la probabilidad de distribuci\'on de una caracter\'istica concreta.
\item Creaci\'on y entrenamiento de modelos de aprendizaje autom\'atico: diseño, programaci\'on y entrenamiento de modelos basados en aprendizaje autom\'atico
\item Visualizaci\'on de datos: representaci\'on gr\'afica de datos para visualizar con claridad patrones, tendencias, interdependencias, etc.
\citep{DigJupyter}
\end{itemize}

\subsection*{An\'alisis exploratorio de datos}
Ayudan a organizar la informaci\'on que nos dan los datos de manera de detectar alg\'un patr\'on de comportamiento as\'i como tambi\'en apartamentos importantes al modelo subyacente.
\citep{OreAnaliExplo}
Su objetivo es explorar, describir, resumir y visualizar la naturaleza de los datos recogidos en las variables aleatorias del proyecto o investigaci\'on de inter\'es, mediante la aplicaci\'on de t\'ecnicas simples de resumen de datos y m\'etodos gr\'aficos sin asumir asunciones para su interpretaci\'on.
\citep{HelAnaliExplo}

\subsection*{Curacion de datos}
Consiste en seleccionar informaci\'on relevante que ya est\'a publicada en Internet, filtrarla, organizarla, añadir un valor adicional y difundirla a nuestra comunidad.
\citep{DuroCur}
Su funci\'on es hacer que la informaci\'on sea \'util y tomar las medidas necesarias para preservarla.
\citep{ItCur}

\section{Cuentas}
\begin{itemize}
\item LastPass
\item GitHub
\item GitKraken
\item Gravatar (WordPress)
\item Trello
\item Bitbucket
\end{itemize}

\bibliography{../references/herramientas} 
\bibliographystyle{apalike}

\end{document}
