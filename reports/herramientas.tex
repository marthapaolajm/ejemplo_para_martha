\documentclass[12pt,letterpaper]{article}
\usepackage[utf8]{inputenc}
\usepackage[spanish]{babel}
\usepackage{amsmath}
\usepackage{amsfonts}
\usepackage{amssymb}
\usepackage{makeidx}
\usepackage{graphicx}
\usepackage{lmodern}
\usepackage[left=2.5cm,right=2.5cm,top=2.5cm,bottom=2.5cm]{geometry}
\thispagestyle{empty}
\begin{document}
%%%%%%%%%%%%%%%%%%%%%%%%%%
\title{\textbf{Tarea 0: 
Definiciones.}}
\author{Martha Paola Jim\'enez Mart\'inez \\
\small{martha.paola.jimenez.martinez@uabc.edu.mx} \\
\small{martha.jimenez@islas.org.mx}}
\date{ \small{Asignatura: PVVC 2021-1}
} 
\maketitle

%%%%%%%%%%%%%%%%%%%%%%%%%%%%%%%%%%%%%%%%%%%%%%
\section*{Herramientas}
\subsection*{Git}
Git es una herramienta que realiza una funci\'on del control de versiones de c\'odigo de forma distribuida, y no depende de un repositorio central.
Este funciona con ramas las cuales est\'an destinadas a provocar proyectos divergentes de un proyecto principal, para hacer experimentos o para probar nuevas funcionalidades.
 \cite[( Rubio, J. 2020)]{ref1}.
Sirve para que la copia de trabajo del c\'odigo de cada desarrollador sea tambi\'en un repositorio que puede albergar el historial completo de todos los cambios.
 \cite[(Bitbucket. s.f.)]{ref2}.

\subsection*{Bash}
Bash es un int\'erprete de comandos que ejecuta, una por una, las instrucciones introducidas por el usuario o contenidas en un script y devuelve los resultados. En otras palabras, act\'ua como interfaz entre el kernel Linux y los usuarios o programas del modo texto.
 \cite[(Canepa, G. 2018)]{ref3}.

\subsection*{Docker}
Es una tecnolog\'ia de creación de contenedores que permite la creaci\'on y el uso de contenedores de Linux.
Sirve para usar los contenedores como m\'aquinas virtuales extremadamente livianas y modulares. Adem\'as, obtiene flexibilidad con estos contenedores: puede crearlos, implementarlos, copiarlos y moverlos de un entorno a otro, lo cual le permite optimizar sus aplicaciones para la nube.
 \cite[( RedHat. s.f.)]{ref4}.

\subsection*{GitHub}
GitHub es un sistema de gesti\'on de proyectos y control de versiones de c\'odigo, as\'i como uno de los repositorios online m\'as grandes de trabajo colaborativo en todo el mundo.
 \cite[(Hostinger Tutoriales. 2019)]{ref5}.
Sirve como un servicio de hosting de repositorios almacenados en la nube. Esencialmente, hace que sea m\'as f\'acil para los individuos y equipos usar Git como la versi\'on de control y colaboraci\'on.
\cite[(Kinsta. 2020)]{ref6}.

\subsection*{GitFlow}
Es un flujo de trabajo aplicado a un repositorio Git. 
Sirve para proyectos que lleven una planificaci\'on de entregas iterativas. Permite la paralelización del desarrollo mediante ramas independientes para la preparaci\'on, mantenimiento y publicaci\'on de versiones del proyecto as\'i como soporta la reparaci\'on de errores en cualquier momento.
\cite[(Claventy. 2020)]{ref7}.

\subsection*{Python}
Python es un lenguaje interpretado, es decir, que no conlleva compilar (scripting), independiente de plataforma y orientado a objetos. 
\cite[(Desarrollo Web. 2003)]{ref8}.
Sirve para interpretar el c\'odigo del programador, donde a su vez lo traduce y ejecuta. Y utiliza diversos modelos de desarrollo, por lo que no exige un estilo \'unico de programaci\'on.
\cite[(Angeles, J. 2020)]{ref9}.


\subsection*{Jupyter Notebook}
Es una interfaz web de c\'odigo abierto que permite la inclusi\'on de texto, v\'ideo, audio, im\'agenes as\'i como la ejecuci\'on de c\'odigo a trav\'es del navegador en m\'ultiples lenguajes. 
\cite[(Cabrera, E. y Diaz, E. s.f.)]{ref10}.
Se utiliza para: 
\begin{itemize}
\item Depuraci\'on de datos: distinguir entre los datos que son importantes y los que no lo son al ejecutar un an\'alisis de big data.
\item Modelizaci\'on estad\'istica: m\'etodo matem\'atico para estimar la probabilidad de distribuci\'on de una caracter\'istica concreta.
\item Creaci\'on y entrenamiento de modelos de aprendizaje autom\'atico: diseño, programaci\'on y entrenamiento de modelos basados en aprendizaje autom\'atico
\item Visualizaci\'on de datos: representaci\'on gr\'afica de datos para visualizar con claridad patrones, tendencias, interdependencias, etc.
\cite[(Digital Guide IONOS. s.f.)]{ref11}.
\end{itemize}

\subsection*{An\'alisis exploratorio de datos}
Ayudan a organizar la informaci\'on que nos dan los datos de manera de detectar alg\'un patr\'on de comportamiento as\'i como tambi\'en apartamentos importantes al modelo subyacente.
\cite[(Orellana, L. s.f.)]{ref12}.
Su objetivo es explorar, describir, resumir y visualizar la naturaleza de los datos recogidos en las variables aleatorias del proyecto o investigaci\'on de inter\'es, mediante la aplicaci\'on de t\'ecnicas simples de resumen de datos y m\'etodos gr\'aficos sin asumir asunciones para su interpretaci\'on.
\cite[(Helix Bios. s.f.)]{ref13}.

\subsection*{Curacion de datos}
Consiste en seleccionar informaci\'on relevante que ya est\'a publicada en Internet, filtrarla, organizarla, añadir un valor adicional y difundirla a nuestra comunidad.
\cite[(Duro, S. 2017)]{ref14}.
Su funci\'on es hacer que la informaci\'on sea \'util y tomar las medidas necesarias para preservarla.
\cite[(It User. 2020)]{ref15}.

\section{Cuentas}
\begin{itemize}
\item LastPass
\item GitHub
\item GitKraken
\item Gravatar (WordPress)
\item Trello
\item Bitbucket
\end{itemize}

\begin{thebibliography}{99}

\bibitem{ref1} Rubio, J. (2020). Qu\'e es Git y para qu\'e sirve. https://openwebinars.net/blog/que-es-git-y-para-que-sirve/

\bibitem{ref2} Bitbucket. (s.f.). Qué es Git. https://www.atlassian.com/es/git/tutorials/what-is-git

\bibitem{ref3} Canepa, G. (2018). Qué es Bash. https://blog.carreralinux.com.ar/2018/04/que-es-bash-shell-interprete-y-mas/

\bibitem{ref4} RedHat. (s.f.) ¿Qué es Docker? https://www.redhat.com/es/topics/containers/what-is-docker

\bibitem{ref5}  Hostinger Tutoriales. (2019) ¿Qué es GitHub y para qué se utiliza? https://www.hostinger.mx/tutoriales/que-es-github/

\bibitem{ref6} Kinsta. (2020) ¿Qué es GitHub? https://kinsta.com/es/base-de-conocimiento/que-es-github/

\bibitem{ref7} Claventy. (2020). Qué es Git flow y cómo funciona. https://cleventy.com/que-es-git-flow-y-como-funciona/

\bibitem{ref8} Desarrollo Web. (2003). Qué es Python. https://desarrolloweb.com/articulos/1325.php

\bibitem{ref9} Angeles, J. (2020) Python. https://www.crehana.com/mx/blog/web/que-es-python/\#que-es-y-para-que-sirve-python 

\bibitem{ref10} Cabrera, E. y Diaz, E. (s.f.) Manual de uso de Jupyter Notebook para aplicaciones docentes. 
https://eprints.ucm.es/id/eprint/48304/1/ManualJupyter.pdf

\bibitem{ref11} Digital Guide IONOS. (s.f.)Jupyter Notebook: documentos web para análisis de datos, código en vivo y mucho más. https://www.ionos.mx/digitalguide/paginas-web/desarrollo-web/jupyter-notebook/

\bibitem{ref12} Orellana, L. (s.f.). Estadística descriptiva o Análisis exploratorio de datos. http://cms.dm.uba.ar/academico/materias/verano2015/estadisticaQ/descriptiva.pdf

\bibitem{ref13} Helix Bios. (s.f.). Análisis exploratorio de datos y visualización de la información. http://www.helixbios.com/analisis-exploratorio-de-datos

\bibitem{ref14} Duro, S. (2017). ¿Qué es la Curación de Contenidos y qué herramientas usar? https://www.webempresa.com/blog/curacion-de-contenidos.html

\bibitem{ref15} It User. (2020). Qué es un curador de datos y por qué es importante. https://discoverthenew.ituser.es/predictive-analytics/2020/06/que-es-un-curador-de-datos-y-por-que-es-importante

\end{thebibliography} 

\end{document}
